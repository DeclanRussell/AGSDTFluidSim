\section*{Cuda Smoothed Particle Hydrodynamics Report }

 \section*{Introduction }

The contents of this report follows the design and implimentation of a real time fluid simulation. This implimentation takes the form of a 3\-D Langrangian grid and takes advanteges of the speed of Nvidia's Cuda A\-P\-I to create a fast realistic fluid simulation. In the following sections you will find an explenation of maths used, optimisations made and the implimentation of this artifact.

\section*{Smoothed Particle Hydrodynamics and Fluid Theory }

When implimenting fluid simulations there is an array of techniques in which you can use. Each of which have there own advantages and disadvanteges. The most prominent of these techniques are Eulerian and Langrangian.
\begin{DoxyItemize}
\item Eulerian Method Looks at fluid motion through specific locations in space. Space is devided up into cells which store attributes about the fluid in that location such as pressure, velocity and desity etc...
\begin{DoxyItemize}
\item Advanteges
\begin{DoxyItemize}
\item Performance determined by grid size not number of particles
\item Fast
\end{DoxyItemize}
\item Disadvanteges
\begin{DoxyItemize}
\item Detail contrained to grid size
\item Simulation size limited to grid size
\end{DoxyItemize}
\end{DoxyItemize}
\item Langrangian Method Focuses on individual particles of the fluid. Each particle stores attributes about its own pressure, velocity and density.
\begin{DoxyItemize}
\item Advanteges
\begin{DoxyItemize}
\item Simulation size not limited.
\end{DoxyItemize}
\item Disadvanteges
\begin{DoxyItemize}
\item Performance tied to number of particles
\item For a realistic simulation we need to have a lot of particles
\end{DoxyItemize}
\end{DoxyItemize}
\end{DoxyItemize}

For this implimentation I will be following the Langrangian method.

\subsection*{Navier Stokes Equations }

Navier stokes equations mathematically model the behaviour of fluid. The they look at calculating forces apparent in the fluid and give us an acceleration with from the sum all these forces. Bellow is the Langrangian fluid equation for weakly compressible flow.\par
 \[ \rho\frac{du}{dt}=-\bigtriangledown\rho+\mu\bigtriangledown^2 u + f \]\par
 On the left of this equation you will see the greek symbol $\rho$ which stands for the density of the of the fluid. This is multiplied by $\frac{du}{dt}$ which is the excelleration of our particle at the next timestep. Now lets break down the three componts on the right hand side of the equation.\par
 \[ -\bigtriangledown\rho \]\par
 This term represents the gradient of pressure of our particle. Giving the direction that the pressure is in and the magnitude of this pressure\par
 \[ \mu\bigtriangledown^2 u \]\par
This term represents the viscosity force acting upon our particle. $\mu$ Being a scaler coefficient that can be set by the user to increase of decrease the viscosity force acting upon our fluid.\par
 \[ f \]\par
Our third and final term in this formulae represents any external forces acting upon our fluid. This could be anything from gravity, suface tension or any other forces you may need to interact with our fluid.

\subsection*{Smoothed Particle Hydrodynamics }

Smoothed particle hydrodynamics is a technique that provides a collection of approximation formulae to solve our Navier Stokes equations. It focuses on scaling the forces in our Navier Stokes equation based on the distance between particles. Closer particles will have a much higher influence on our forces than particles further away. To do this we will specify a wighting kernel which I will specify later.

\subsection*{Calculating our forces }

\subsection*{Density }

As you can see from our Navier Stokes equations every force must be devided by the particles density to calculate the acceleration therefore the first force we must calculate must be the density of our particles. To do this we use the formulae bellow.\par
 \[ \rho(x_i)= \sum\limits_{j}m_j W_{default}(x_i-x_j,h) \]\par
 This formulae represents the sum of $m$ which is the mass of our neightbouring particle. In our simluation our mass will be a constant with all particles having the same mass. This is then multiplied by function $W$ which is our weighting kernel. More on how we calculate that later.

\subsection*{Pressure Gradient }

Once we have our the densities of our particles calulated we can now move on to computing our pressure gradient. To do this firstly however we must calculate the pressure per particle in our simulation. This is computed with the following equation,\par
 \[ p_i = k(\rho_i - \rho_0) \]\par
 Here we have variable $k$ which is the gas constant of our fluid and will be set by the user depending on their desired fluid behaviour. $\rho_i$ is the density of the particle that we are calcuating the pressure for and $\rho_0$ is the rest density of our fluid. This is also a constant that the use will set by the user depending on their desired fluid behaviour. Now that we have our pressure per particle, we can use this to to compute our pressure gradient function,\par
 \[ f_i^{pressure} = -\sum\limits_{i\neq j}p_j\frac{m_j}{\rho_j}\bigtriangledown W_{pressure}(x_i-x_j,h) \]\par
The formulae above represents the summation of particle properties $j$ when $j$ is not equal to $i$. The variable $m_j$ is our particle mass, $\rho_j$ is the density of our particle which we calculated using the equation in the previous section and $W$ is our pressure weighting kernel which I will discuss later. However there is a problem when using this equation. The pressure force calculated is not symetrical as particle $i$ only uses the the pressure at particle $j$ to compute the pressure gradient. This problem can be solved however by using an different proposed equation \mbox{[}M\-J92\mbox{]}, \[ f_i^{pressure} = -\sum\limits_{i\neq j}(\frac{p_i}{\rho_i^2} + \frac{p_j}{\rho_j^2})m_j\bigtriangledown W_{pressure}(x_i-x_j,h) \]

\subsection*{Viscosity Force }

Next we can calculate our viscosity force, the larger this force is the more it gives the fluid a thicker or stickier appearence. An example of a very viscous fluid would be syrup. To calculate this force we use the following equation, \[ f_i^{viscosity} = \mu\sum\limits_{i\neq j}(u_j-u_i)\bigtriangledown^2 W_{viscosity}(x_i-x_j,h) \]\par
In this equation we have one more unknown $\mu$, this represents the viscosity force. This is a scaler value set by the user to for the desired visocity influence on the fluid. Also you may notice that this formulae also contains a third weighting kernel which will be explained in the next section.

\subsection*{Other forces }

Finally our final unknown is $f$. For this implimentation $f$ will only represent gravity which we will assume is a constant 9.\-8m/s in the negative y direction. However for future work this can be extented to also include any other forces you may want to include such as surface tension or some kind of interaction forces.

\subsection*{Weighting kernels }

As mentioned in the previous sections all our force equations use a weighting kernel which scales the force value calculated based on the distance particles are from each other. The first of these kernels we encountered was the function $W_{default}$ which can be denoted,\par
 \[ \newcommand{\twopartdef}[4] { \left\{ \begin{array}{ll} #1 & \mbox{if } #2 \\ #3 & \mbox{if } #4 \end{array} \right. } W_{default}(r,h) = \frac{315}{64\pi h^9}\twopartdef{(h^2-||r||^2)^3}{0\leq ||r|| <h}{0}{||r||>h} \]

In this equation $r$ is equivilent to vector from our neighbour particle to our current particle. The letter $h$ is the smoothing length of our equation. This variable is set before the simulation is run and will affect the behaviour of our fluid.\par
The above weighting kernel will look like this when assuming that the soomthing length is 1,\par
 This smoothing kernel is sutible for use on our density calculations but however will not be suitible for when we calculate our pressure. This is due to the gradient of the function tending to 0 as the distance of our particles tends to 0. To use this weighting kernel would create clustering of our particles. Ideally we would want our pressure kernel to continually get larger as our distance approaches 0 and is at the point of highest pressure. This is solved in \mbox{[}M\-M03\mbox{]} with their proposed \char`\"{}\-Spikey\char`\"{} kernel. \[ W_{pressure}(r,h) = W_{spikey}(r,h) = -\frac{45}{\pi h^6} \frac{r}{||r||}(h-||r||)^2 \]

Again assuming that our smoothing length $h$ is 1 our smoothing kernel will look like,\par
 Notice how our kernel now tends to infinity due to our division of the length of $r$ solving our clustering problem.

Our final weighting kernel is for our viscosity term. Also proposed \mbox{[}M\-M03\mbox{]} is the following kernel,\par
 \[ W_{viscosity}(r,h) = -\frac{45}{\pi h^6}r(h-||r||) \] If is important to note that unlike the pressure weighting kernel, the values from our viscosity weighing are always positive. This is because the viscosity term acts as a damping force, if values were to become negative it would increase the energy of our particles.\par
Again assuming that our smoothing length $h$ is 1 our smoothing kernel will look like,\par
 \subsection*{Integration Methods }

\subsection*{Euler }

To update our particles position we must integrate our acceleration to calculate first our velocity and then our displacement. The most basic integration we can use is Euler integration which goes as follows,\par
 \[ u = u + adt \] \[ x = x + udt \]

We have four unknowns in these equations. $x$, $u$ and $a$ are our position, velocity and acceleration. Our final unknown $dt$ is the timestep of our update. However the problem with this method is that as the simulation the chance of error when using these equations increases over time. These means the longer the simulation is running for the more inaccurate and unstable it will get. \subsection*{Leap Frog }

A more stable method to use is the Leap Frog method. This method uses future half step velocity and a previous half step velocity to calculate the position at the next half time step. This method proves much more stable than our Euler method. It is achieved with the equations below,\par
 \[ u_{t+\frac{1}{2} \triangle t} = u_{t- \frac{1}{2} \triangle t} + \triangle t a_t \] \[ u_{t-\frac{1}{2} \triangle t} = u_0 - \frac{1}{2}\triangle t a_0 \] \[ x_{t+\frac{1}{2} \triangle t} = x_t - \triangle tu_{t+\frac{1}{2} \triangle t} \] Below is a graph comparing the error of Leap Frog compared to other popular integration techniques over time with a deliberately high timestep used.\par
As you can see from the graph although the accuracy may drop the integration technique stays very stable over long periods of time.

\section*{Optimisations }

\subsection*{Spacial Hash }

When sampling neighbouring particles it it computationally inificient to sample all the particles in our scene. If every particle $n$ samples every orther particle $n$ then we reach an overall computational complexity of $O(n^2)$ which means our simulation will get exponetially slower the more particles we simulate in our scene. Any particles outside our smoothing length are given an influence of 0 on our particle, this means we can exclude them from our calculations. Idealy we want to only sample a set number of particles within our smoothing length which we can accomplish this with the use of a spacial hash. This spacial hash will assign particles that are near each other a unique key. we can then use this key to identify the particles that we need to sample. If we choose the maximum number of particles to sample and keep that as a constant value this reduces our complexity to $O(n)$ which is fast. For this implimentation I have used a very simple hash function,\par
 \[ p_n = p/s_g \] \[ r = s_g/h \] \[ p_g = \lfloor p_n*r \rfloor \] \[ idx = p_{gx}*r^2 + p_{gy}*r +p_{gz} \] Where $p$ is our position, $s_g$ is the dimention size of our hash grid and $h$ is our smoothing length. $r$ refers to the resolution of our grid. We compute it this way so that every cell in our hash is the size of our smoothing length and particles of this cell and neighbouring cells are likely to be within our smoothing length. This is visualised in the image bellow.\par
As you can see from the image above, neighbouring cells particles must be taken into account in our samples as they may still lie within the smoothing length of our particle. Finding neighbouring hash keys is fast though and can be computed in the following way,\par
 \[ neighbour idx = idx + (dx*r^2 + dy*r + dz) \] Where $dx,dy$ and $dz$ are the offset of the neighbour hash cell we desire.

\subsection*{C\-U\-D\-A }

C\-U\-D\-A or Compute Unified Device Architecture is a parallel computing A\-P\-I create from Nvidia which allows you to take control of your G\-P\-U's parallel architecture. This is ideal for our fluid simulation as it gives us the ability to process all our particle calculations at the same time rather than if we were to do this on the C\-P\-U in serial. This gives us the ability to simulate a much larger number of particles in our simulation. In the following sections I will discuss the implimentation of this.

\subsection*{C\-U\-D\-A implimentation }

Important things to take into account when programming on the G\-P\-U.\par

\begin{DoxyItemize}
\item Copying from G\-P\-U to C\-P\-U is slow, preferably we want to keep everything on the G\-P\-U as much as possible.
\item Accessing global memory from the G\-P\-U is slow, we should keep this to a minumum as much as possible. \par
First of all we need to set up some buffers to make our implimentation posible. The buffers needed are
\item Position buffer, size of how many particles we have.
\item Velocity buffer, size of how many particles we have.
\item Hash key buffer, size of how many particles we have.
\item Cell Occupancy buffer, size of our hash table. This can be calculated from the resolution cubed.
\item Cell Index buffer, size of our hash table. \par

\end{DoxyItemize}
\begin{DoxyEnumerate}
\item Hashing partilces\par
 The first thing we need to do is hash our particles to give them a key based on their position. To do this we assign a thread to for everyparticle. Now simply use the equation from the previous section to out put our hash keys into a new buffer.
\item Sort our particle data by hash key.\par
 This is important to reduce banking conflicts when accessing from the global memory of our G\-P\-U. Global memory copies are slow so we want to do as little of them as possible. When you tell a cuda kernal to access global memory the bus will also give you the data from contiguous memory locations aswell. So for speed improvements we need to keep our memory as contiguous as possible reducing the number of global memory copies we do. To achieve this sort we can simply use thrust, a library of prewritten cuda functions. Thrust has its own sort\-\_\-by\-\_\-key function which is perfectly suitible for our needs.
\item Count Cell Occupancy\par
 We need to know the cell occupancy to know how many particles are in each cell. This also helps use identify where our particles are stored in memory. To count the cell occupancy is fairly simple. We assign a thread for every entry in our hash key buffer and we increment the value in our Cell Occupancy buffer relative to this hash key. However it is important to know that to avoid race conditions between threads. To solve this we can use C\-U\-D\-A's atomic add function to do this. This uses a mutex to lock the memory address while it is being modified by the current thread.
\item Create our cell index buffer\par
 This will give us the particle index of particles that belong in our cell. To compute this we can just create a running total of all the entries in our cell occupancy buffer. Thrust can again be used as its exlusive\-\_\-scan function does just this in paralell.
\item Fluid Solving Now we have all our buffers prepared we can use our navier stokes equations and solve for our new particle position.
\begin{DoxyEnumerate}
\item Assign a block of threads to every cell in our hash table.
\item Load our neighbouring particle data into shared memory.\par
 As we need to access neighbour particle data a lot in our calculations accessing global memory multiple times is going to add a lot of overhead to our solver. In the G\-P\-U architecture however every block has its own shared memory which can be accessed by all threads in that block. This shared memory is considerably faster than accessing global memory. Therefore its more efficient for us to copy once from global memory into shared memory than continuously copy from global memory.
\item Assign every thread a particle.
\item Perform our calculations on each particle
\item Update particle positions
\end{DoxyEnumerate}
\item Collision detection\par
 The collision detection in this implimentation is very simple. I simply assign every particle to a thread and perform A\-A\-B\-B collision detection on every position. If it has collided with the collision object the position is simply pushed back and the velocity of the particle is flipped.
\end{DoxyEnumerate}

\subsection*{Reducing some arithmetic computation }

Threads on the G\-P\-U do not have as much arithmetic power when compared to the C\-P\-U. This means that arithmetic operations that we perfrom on the G\-P\-U will have an impact on overhead on each kernal launch. To improve this we want to limit the number of operations as much as we can. A good example of this is our weighting kernals, \[ \newcommand{\twopartdef}[4] { \left\{ \begin{array}{ll} #1 & \mbox{if } #2 \\ #3 & \mbox{if } #4 \end{array} \right. } W_{default}(r,h) = \frac{315}{64\pi h^9}\twopartdef{(h^2-||r||^2)^3}{0\leq ||r|| <h}{0}{||r||>h} \] Notice that during our simulation assuming that we dont regularly change our smoothing length then $\frac{315}{64\pi h^9}$ will remain constant. It actually become more efficient if we precalculate this on the C\-P\-U and load it in as a paramiter to our C\-U\-D\-A kernal. We can see another example in our pressure kernal,\par
 \[ f_i^{pressure} = -\sum\limits_{i\neq j}(\frac{p_i}{\rho_i^2} + \frac{p_j}{\rho_j^2})m_j\bigtriangledown W_{pressure}(x_i-x_j,h) \] Throughout our sum loop $\frac{p_i}{\rho_i^2}$ will be the same. Instead of recomputing this every loop we can just compute it once, store the value and use it when needed. Secondly our mass $m_j$ is constant, this means we can move it out side our loop reducing our multiply operations to just 1 time rather than $j$ times. Our final equation will look like this, \[ p1 = \frac{p_i}{\rho_i^2} \] \[ f_i^{pressure} = -m_j\sum\limits_{i\neq j}(p1 + \frac{p_j}{\rho_j^2})\bigtriangledown W_{pressure}(x_i-x_j,h) \] Simplifications like this can be made to all our equations. \subsection*{C\-U\-D\-A Streams and Multiple Simulations }

Another optimisation that C\-U\-D\-A provides are streams. This allows you to launch C\-U\-D\-A operations with a chance of operations in different streams being run concurrently. For this implimentation I have used this when creating seperate fluid simulations at the same time. This technique could be explored in further research to improve on how we launch the kernals to update our fluid simulation. More information about C\-U\-D\-A streams here\-: \href{http://on-demand.gputechconf.com/gtc-express/2011/presentations/StreamsAndConcurrencyWebinar.pdf}{\tt http\-://on-\/demand.\-gputechconf.\-com/gtc-\/express/2011/presentations/\-Streams\-And\-Concurrency\-Webinar.\-pdf} \subsection*{Simulation Performance }

Bellow you can see a graph representing the simulation performance when simulating different quantities of particles. 

\section*{Rendering }